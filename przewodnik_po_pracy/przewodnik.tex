% By zmienic jezyk na angielski/polski, dodaj opcje do klasy english lub polish
\documentclass[polish,12pt]{aghthesis}
\usepackage[utf8]{inputenc}
\usepackage{url}

\author{Grzegorz Miejski, Maciej Prokopiuk}

\title{Aplikacja webowa do wizualizacji danych na osi czasu}

\supervisor{mgr inż. Kamil Piętak}

\date{2015}

% Szablon przystosowany jest do druku dwustronnego, 
\begin{document}

\maketitle



\section{Cel prac i wizja produktu}
%\section{Project goals and vision}
\label{sec:cel-wizja}
\emph{Charakterystyka problemu, motywacja projektu (w tym przegld
  istniejcych rozwiązań prowadzca do uzasadnienia celu prac), oglna
  wizja produktu, krtkie studium wykonalnoci i analiza zagrożeń.}


Zgodnie z tematem projektu, problemem jest wizualizacja zdarzeń na osi czasu. Rozważano dwa możliwe rozwiązania techniczne. Pierwszym było stworzenie aplikacji, z której użytkownicy mogliby korzystać niezależnie, a drugim stworzenie biblioteki, bądź widgetu, który możnaby dołączyć do dowolnej aplikacji webowej. 
\\

	Zdecydowano się na drugi wariant rozwiązania, który jest zdecydowanie bardziej uniwersalny, elastyczny i łatwiejszy do rozbudowy jak i również do personalizacji przez końcowych użytkowników. 
\\

	Projekt ma więc za zadanie stworzenie biblioteki służącej do wyświetlania informacji o zdarzeniach na osi czasu, którą można w prosty sposób dołączyć do dowolnej aplikacji internetowej. Istnieje wiele gotowych rozwiązań i aplikacji, które umożliwiają proste wyświetlanie danych na osiach czasu, jednak żadne nie spełniało wszystkich postawionych wymagań. Z tego względu zdecydowano się stworzyć własne narzędzie dostosowane do potrzeb klienta.
\\

Początkowo rozważano stworzenie produktu od zera, jednak ze względu na ryzyko związane z ograniczeniem czasowym projektu oraz brakiem doświadczenia zespołu projektowego w tego typu problemach, postanowiono skorzystać z istniejącej już biblioteki Chap-Links, która oferuje największy zasób funkcjonalności spośród wszystkich znalezionych narzędzi, i rozszerzyć ją o brakujące funkcje aby spełnić wszystkie wymagania stawiane przez klienta.
\\

Użytkownikowi zostaną udostępnione funkcję do tworzenia i zarządzania osią czasu, dostępu do danych w niej zawartych oraz prostych narzędzi do analizowania wcześniej wspomnianych danych, t.j. wyszukiwanie konkretnych zdarzeń względem ich zawartości, dat rozpoczęcia i końca, czy znajdowanie zdarzeń, których ramy czasowe się nakładają. Dodatkowo w ramach projektu ma zostać przedstawiona aplikacja demonstrująca przykładowe użycie stworzonej biblioteki.
\\

Dzięki oparciu się na bibliotecie chap-links Timeline istnieją duże szanse na zakończenie się projektu sukcesem. Byliśmy jednak przygotowani na niespodziewane problemy, które mogłyby wynikać z szczególnych realizacji niektórych aspektów biblioteki, które uniemożliwiałyby lub utrudniałyby realizację wybranych funkcjonalności. Mając jednak bezpośredni dostęp do kodu źródłowego biblioteki moglibyśmy dzięki spokojnej pracy wystarczająco wcześnie odkryć te sytuację i spróbować rozwiązać problem zmieniając kod bezpośrednio danej biblioteki.


\section{Zakres funkcjonalności}
%\section{Functional scope}
\label{sec:zakres-funkcjonalnosci}

\emph{Kontekst użytkowania produktu (aktorzy, współpracujące systemy)
  oraz najważniejsze wymagania funkcjonalne i niefunkcjonalne.}

\section{Wybrane aspekty realizacji}
%\section{Selected realization aspects}
\label{sec:wybrane-aspekty-realizacji}

\emph{Przyjęte założenia, struktura i zasada działania systemu,
  wykorzystane rozwiązania technologiczne wraz z krótkim uzasadnieniem
  ich wyboru.}

\section{Organizacja pracy}
%\section{Work organization}
\label{sec:organizacja-pracy}

\emph{Struktura zespolu (role poszczeglnych osób), krótki opis i
  uzasadnienie przyjętej metodyki i/lub kolejności prac, planowane i
  zrealizowane etapy prac ze wskazaniem udziau poszczególnych
  członków zespou, wykorzystane praktyki i narzędzia w zarządzaniu
  projektem.}
\subsection{Zespół projektowy}

Zespół składał się z dwóch osób - Grzegorz Miejski oraz Maciej Prokopiuk.
\\

W pierwszych tygodniach projektu zastanawiano się jak podzielić zakres obowiązków, jednak nie było do końca wiadomo jak będzie przebiegała praca. Jednym z pomysłów było podzielenie się na stronę back-end`ową i front-end`ową, ale kiedy wyklarowało się, że głównym celem projektu jest stworzenie widgetu, bądź biblioteki umożliwiającej użytkownikom wizualizację danych zdecydowano się porzucić ten pomysł. 
\\

Po utworzeniu wstępnego backlogu zdecydowano się dzielić implementacją poszczególnych funkcjonalności. Jeden programista był odpowiedzialny za określoną część biblioteki np. dodawanie zdarzeń, wyszukiwanie zdarzeń, znacznik elementów, których czas trwania zawierał się w czasie wyspecyfikowanym przez użytkownika.
\\

Ze względu na małą ilość osób w zespole nie było podziału ról i obowiązków. Obie osoby zajmowały się zarówno programowaniem, jak i zarządzaniem projektem.  
\\

\subsection{Przyjęta metodyka pracy}

Ustalono, że proces będzie prowadzony z zachowaniem pewnych zasad metodyki Agile. Całość pracy podzieliliśmy na poszczególne iteracje, dzięki czemu mieliśmy ciągly obraz stopnia ukończenia projektu w danej chwili oraz mogliśmy prezentować klientowi poszczególne funkcjonalności po każdej z nich, dzięki czemu od razu zbieraliśmy opinię na temat naszej pracy. Wewnątrz iteracji trwających od miesiące do dwóch miesięcy pracowano wzorując się na metodyki Kanban, polegającej na priorytetyzowaniu zadań i wykonywaniem ich w kolejności zgodnej z ustalonym priorytetem. Datą rozpoczęcia projektu ustanowiono dzień 17 marca 2014 roku. Wtedy miało miejsce pierwsze spotkanie z Pracowni Projektowej, na której poruszana była kwestia prowadzenia projektu i wymagań z tym związanych. Następnie proces pracy podzielono na 5 iteracji. Jako daty rozpoczęcia i zakończenia danych iteracji ustalono spotkania z klientem w celu przedstawienia zrealizowanych rzeczy. Oto jak ukształtował się podział iteracji: 
\\\\

 TODO - co kiedy zrobiliśmy i kto?
\\\\

\subsection{Użyte narzędzia}

Przy reaizacji projektu posłużyliśmy się wieloma narzędziami, które ułatwiały nam prowadzenie projektu i wzajemną współpracę w zespole.
\\

\subsubsection{Zarządzanie projektem}

Do zarządzania backlogiem użyto serwisu Trello - http://trello.com
\\

Jest to darmowe, bardzo elastyczne narzędzie umożliwiające w prosty sposób zarządzać projektem i organizować pracę. Udostępnia bardzo intuicyjny interfejs użytkownika pomagający szybko przenosić bądź grupować historyjki. Pozwala również na przypisywanie konkretnych osób do określonych zadań, dzięki czemu można bardzo szybko dowiedzieć się kto aktualnie nad czym pracuje.
\\

Trello świetnie sprawdził się w swojej roli. W trakcie trwania projektu ujmowano z backlogu pewne zadania, które okazały się na przykład drugoplanowe lub zupełnie niepotrzebne, a także dodawano nowe zadania wynikające z przebiegu projektu. Wszystko przebiegało bardzo sprawnie, a członkowie projektu w jasny sposób mogli dowiedzieć się co aktualnie się dzieje w projekcie, dzięki podziałowi projektu na trello na kolejne iteracje oraz grupowaniu zadań jeszcze nie zaczętych bądź chociażby duplikowanych. 
\\

\subsubsection{Zarządzanie kodem}

Do zarządzania kodem zdecydowano się użyć narzędzia Git (http://git-scm.com/ ) wraz z repozytorium kodu na platformie GitHub (http://github.com). Git został wybrany ze względu na rozproszony system pracy. Każdy mógł pracować na własnej lokalnej kopii kodu oraz na branchach, umożliwiających łatwiejsze zarządzanie funkcjonalnościami. Dzięki temu programiści mogli niezależnie organizować sobie pracę. Git posiada wiele przydatnych i przyśpieszających prowadzenie projektu funkcji, jak na przykład rebase, umożliwiający zachowanie lepszej historii napisanego kodu, bądź automatyczne merge branchy do głównej gałęzi projektu (jeśli nie wynikają jakieś konflikty, których Git sam nie może rozwiązać). Repozytorium jest publiczne, więc każdy może zajrzeć w tworzony kod, dodawać do niego komentarze, bądź tworzyć pull-requesty, w celu poprawy części kodu, bądź dodania nowych funkcjonalności. 

\subsubsection{Komunikacja}

W przypadku, gdy niemożliwe było spotkanie pomiędzy programistami na żywo korzystano z Google Hangouts (http://www.google.com/+/learnmore/hangouts/?hl=pl). Jest to narzędzie umożliwiające tworzenie prostych w obsłudze wideokonferencji dla wielu osób. Dodatkowym atutem tego narzędzia jest możliwość udostępniania ekranu uczestnikom konferencji, dzięki czemu można rozmawiać o problemach pokazując od razu kod, który dotyczy danego zagadnienia. Używano tego narzędzia zarówno do inspekcji kodu jak i ustalania spraw bieżących związanych z projektem. 



\section{Wyniki projektu}
%\section{Project results}

\label{sec:wyniki-projektu}

\emph{Najważniejsze wyniki (co konkretnie udało się uzyskać:
  oprogramowanie, dokumentacja, raporty z testów/wdrożenia, itd.)
  i ocena ich użyteczności (jak zostało to zweryfikowane --- np.\ wnioski
  klienta/użytkownika, zrealizowane testy wydajnościowe, itd.),
  istniejące ograniczenia i propozycje dalszych prac.}

% o ile to mozliwe prosze uzywac odwolan \cite w konkretnych miejscach a nie \nocite

\nocite{artykul2011,ksiazka2011,narzedzie2011,projekt2011}

\bibliography{bibliografia}

\end{document}
